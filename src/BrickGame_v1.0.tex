\documentclass{article}
\usepackage{array}
\usepackage{geometry}
\geometry{a4paper, margin=1in}

\usepackage{fontspec}
\setmainfont{DejaVuSans}

\title{Описание программы BrickGame v1.0 aka Tetris}
\author{Quinting}
\date{\today}

\begin{document}

\maketitle

\section{Введение}

Цель игры \textbf{BrickGame v1.0 aka Tetris} в наборе очков за построение линий
\newline из генерируемых игрой блоков. Очередной блок, сгенерированный игрой, 
\newline начинает опускаться вниз по игровому полю, пока не достигнет нижней 
\newline границы или не столкнется с другим блоком. Пользовать может поворачивать
\newline фигуры и перемещать их по горизонтали, стараясь составлять ряды. После 
\newline заполнения ряд уничтожается, игрок получает очки, а блоки, находящиеся 
\newline выше заполненного ряда опускаются вниз. Игра заканчивается, когда 
\newline очередная фигура останавливается в самом верхнем ряду.

\section{Конечный автомат игры (состояния): }
\begin{itemize}
    \item NONE — состояние, в котором игра ждет, пока игрок нажмет кнопку
    \newline готовности к игре.
    \item SPAWN — состояние, в которое переходит игра при создании очередного
    \newline блока и выбора следующего блока для спавна.
    \item MOVE — основное игровое состояние с обработкой ввода от поль-
    \newline зователя — поворот блоков/перемещение блоков по горизонтали.
    \item SHIFT — состояние, в которое переходит игра после истечения таймера. 
    \newline В нем текущий блок перемещается вниз на один уровень.
    \item ATTACH — состояние, в которое преходит игра после «соприкосновения»
    \newline текущего блока с уже упавшими или с землей. Если образуются запол-
    \newline ненные линии, то она уничтожается и остальные блоки смещаются вниз. 
    \newline Если блок остановился в самом верхнем ряду, то игра переходит в 
    \newline состояние «игра окончена».
    \item GAME OVER — игра окончена.
\end{itemize}

\section{Управление игры кнопками}
Реализована поддержка всех кнопок, предусмотренных на физической консоли:
\begin{itemize}
    \item 's' -> начало игры,
    \item 'p' -> пауза,
    \item 't' -> завершение игры,
    \item Стрелка влево — движение фигуры влево,
    \item Стрелка вправо — движение фигуры вправо,
    \item Стрелка вниз — падение фигуры,
    \item Стрелка вверх — не используется в данной игре,
    \item Пробел -> действие (вращение фигуры).
\end{itemize}

\section{Подсчет очков, механика уровней и рекорд в игре}
Максимальное количество очков хранится между запусками программы и изменяется
\newline во время игры, если пользователь во время игры превышает текущий 
\newline показатель максимального количества набранных очков.
\begin{itemize}
    \item 1 линия — 100 очков;
    \item 2 линии — 300 очков;
    \item 3 линии — 700 очков;
    \item 4 линии — 1500 очков.
\end{itemize}

Каждый раз, когда игрок набирает 600 очков, уровень увеличивается на 1. 
\newline Повышение уровня увеличивает скорость движения фигур. 
\newline Максимальное количество уровней — 10.

\end{document}
